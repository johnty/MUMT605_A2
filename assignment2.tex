\documentclass[12pt]{article}
\usepackage{amssymb}
\usepackage{amsmath}
\usepackage{float}

%BEGIN for \sha (dirac comb)
\usepackage[OT2,T1]{fontenc}
\DeclareSymbolFont{cyrletters}{OT2}{wncyr}{m}{n}
\DeclareMathSymbol{\sha}{\mathalpha}{cyrletters}{"58}
%END for \sha (dirac comb)

\title{MUMT605 Assignment 2}
\author{Johnty Wang}
\date{14 Nov 2014}

\begin{document}
\maketitle

\section*{Part 1}
\begin{enumerate}


\item[1]{

Given,
\begin{align*}
\omega[k] = \dfrac{Arg X [t,k] - Arg X [s, k] + 2\pi p}{H}\\
\end{align*}

(Intuitive meaning: frequency is given by the phase difference at two time points, plus a potential multiple of 2$\pi$, divided by time between the two points in which this phase difference occurred)

When $t - s = 1 = H$, we have:
\begin{align*}
\omega[k] = Arg X [t,k] - Arg X [s, k] + 2\pi p
\end{align*}

Now, since we're dealing with a sampled signal, we must have 

$-pi < \omega[k] < \pi$ to satisfy the Nyquist criterion and since $t - s = 1$, it is impossible for a signal below this limit to increment more than $2\pi$ in the timespan of \textbf{one single sample} and still be adequately represented without aliasing, so the only possible value of p is $zero$.
}

\item[2]{

As described in the paper, we need a hop size $H \leq \dfrac{N}{2K}$ where $K$ is the bandwidth (in samples) of the window and $N$ is the window size (in samples). Putting in $t - s = H$ and $C_w = K$ and $M$ for window size gives

\begin{align*}
H \leq \dfrac{M}{2C_w}
\end{align*}

(note: I'm not totally certain the definition of $C_w$, I assume its the same as 'K' based on my interpretation of notes taken in class)

This condition can also be satisfied when $t - s$ is 2$\pi$ multiples of a given frequency (for that frequency).
}

\item[3]{

Since $H = u_r-u_{r-1}$ and is the amount of time that a particular frequency $\omega(u_r,k)$ will have evolve for since the previous phase $Arg Y (u_r,k)$, we have simply:

\begin{align*}
\lambda(u_r, k) = Arg Y (u_r,k) +H\omega(u_r,k)
\end{align*}

}

\item[4]{

We still need to maintain the bandwidth-related limit from 2.) above, and in the stretched case, we essentially have a larger gap in the hop given by $\alpha$. Hence, we have the following:
\begin{align*}
\alpha t_r - s_r \leq \dfrac{M}{2C_w}
\end{align*}

}

\item[5]{

}

\item[6]{

Looking at the original equation with $t_{r-1}-s_r  \neq 1$ (in other words, $H \neq 1$:
\begin{align*}
\omega[k] = \dfrac{Arg X [t,k] - Arg X [s, k] + 2\pi p}{t_r-s_r}\\
\end{align*}

}

\item[7]{

}

\item{
For a given function

\begin{align*}
y(t) &= \left\lbrace\begin{array}{ll}
x(t) & t\geq 0\\
0 & \text{elsewhere}
\end{array}\right.
\end{align*}
Then, based on the definition of | |:
\begin{align*}
y(|t|) &= \left\lbrace\begin{array}{ll}
x(t) & t\geq 0\\
x(-t) & t < 0
\end{array}\right. \\
\end{align*} 
Therefore, for this particular example, we have:
\begin{align*}
x(t) &= s(t) + s(-t) \\ \\
\text{from the property of the fourier transform:} \\
\text{if } a(t) \rightarrow A(f) \\
\text{then } a(-t) \rightarrow A(-f) \\ \\
\text{so } x(t) \rightarrow X(f) \\
\implies X(f) &= S(f) + S(-f)\\
\implies  X(f) &= \dfrac{1}{\alpha - 2\pi j f} + \dfrac{1}{\alpha + 2\pi j f}\\
&=  \dfrac{\alpha + 2\pi j f + \alpha - 2\pi j f}{(\alpha - 2\pi j f)(\alpha + 2\pi j f)}\\
&= \dfrac{2\alpha}{\alpha^2 + 4\pi f^2}
\end{align*}
}
\item{
When we sample the spectrum $X(f)$ of the time series signal $x(t)$ using a dirac comb, it creates a ``periodized'' version of the signal, which can be expressed as:

 \begin{align*}
X'(f) &= X(f)\sha_{T_0}(f) \\
\text{where} \\
\sha_{T_0}(f) &= \sum_{k=-\infty}^{\infty} \delta(f - k{T_0}) \\
\implies X'(f) &= \dfrac{2\alpha}{\alpha^2 + 4\pi f^2}  \sum_{k=-\infty}^{\infty} \delta(f - k{T_0})
\end{align*}
}
\item{
From the above, we can see that $X'(f)$ is nonzero where $f = kT_0$, which means the amplitude of the $k$th harmonic can be expressed as:
\begin{align*}
\dfrac{2\alpha}{\alpha^2 + 4\pi k^2{T_0}^2}
\end{align*}
}
\item{
One advantage of this method is that the calculation is very simple - by exploiting the additive and time reversal properties of the fourier transform, we did not need to evaluate the integral for the |t| case.
}

\end{enumerate}

\newpage

\section*{Part 2}

\subsection*{Overview}
For this part, I implemented a matlab function called \verb|A2_func| in Matlab, with its help/description as follows:

\begin{verbatim}
% code here
% code here
\end{verbatim}

The internal code comments provide explanation of the process, but the overall process is as follows:

\begin{itemize}
\item{Compute discrete spectrum}
\item{Generate frequency independent wavetable with correct duty cycle from spectrum, that is otherwise independent of sample rate}
\item{Produce frequency correct wavetable given sample rate, target frequency and fill up entire buffer corresponding to required duration of the synthesized output}
\end{itemize}


\subsection*{Putting it together}
Below is a file listing of the submitted assignment:
\begin{itemize}
\item \verb|A2_func|: the main time stretching function
\item \verb|runme.m|: the tester application that does the following:
\begin{enumerate}
	\item 	Loads sample waveform from file
	\item   Calls the function with a few different values of time stretch
	\item	Plays back original, and time stretched versions
\end{enumerate}
\end{itemize}

\end{document}